\section{Theory}\label{sec:Theory}

\newcommand{\state}{\ket{\psi}}
\newcommand{\hoket}{\ket{\phi}}
\newcommand{\hoep}{\ket{\phi +}}
\newcommand{\hoem}{\ket{\phi -}}
\newcommand{\pin}{\psi_\text{in}}
\newcommand{\pout}{\psi_\text{out}}
\newcommand{\kpin}{\ket{\psi_\text{in}}}
\newcommand{\kpout}{\ket{\psi_\text{out}}}
\newcommand{\mop}{\Omega_{+}}
\newcommand{\mom}{\Omega_{-}}
\newcommand{\melt}[3]{\left|\mel{#1}{#2}{#3}\right|^{2}}
\newcommand{\scat}{\mathcal{S}}
\newcommand{\mscat}{\(\scat\)}
\newcommand{\oop}[1]{\mathcal{#1}}
\newcommand{\moop}[1]{\(\oop{#1}\)}
\renewcommand{\vu}[1]{\mathbf{\hat{\text{$#1$}}}}

\subsection{Effective Field Theory}

The quantum field theory developed in the 1930s and 1940s had
The formalism of computing Feynman diagrams led to difficulties when working
with loops. Infinities pop up left, right and center, giving nonsensical
results. A formalism was developed by Feynman, Schwinger, Tomonaga and Dysin [ref] to nevertheless obtain sensible values
through systematically canceling infinities. Despite the impressive agreement with
experiments [ref], many [who?] felt icky about the method, afraid it was nothing
more but a mathematical trick, and our failure to understand it implied a
failure to understand the physical significance.

It wasn't before 1970s[when?] that Ken Wilson, an otherwise unknown physicist,
illuminated the issue [ref]. The loop diagrams gave infinities because one has to sum
over all possible [diagrams. clumsy structure, rewrite], for all energies. However, we do not know
that QFT is the correct theory for high energy physics, nor is it necessary to
assume so in order to do calculations at lower energies. One can instead set an
arbitrary[not practically arbitrary, make clearer] limit \(\Lambda\) above which
is the high energy, or \textit{ultraviolet} (UV), regime, and below which is the low
energy, or \textit{infrared} (IR), regime [det ble nøstes. nøst opp]. Ken Wilson
argued it is better we admit the physics of UV is unknown and treat it as such,
and instead develop a low energy model. That the physics of UV is irrelevant can
be seen from the fact that any low energy model has many consistent high energy
models, or \textit{completions} [ref]. This gave rise to \textit{effective field
  theories} (EFT).

A more or less general procedure was developed to generate an EFT\cite[p.~7]{Machleidt_2011}:
\begin{enumerate}
\item  Identify the soft and hard scales, and the degrees of freedom appropriate
  for (low-energy) nuclear physics.
\item Identify the relevant symmetries of low-energy QCD and investigate if and how they are broken.
\item Construct the most general Lagrangian consistent with those symmetries and symmetry breakings.
\item Design an organizational scheme that can distinguish between more and less important contributions:
  a low-momentum expansion.
\item Guided by the expansion, calculate Feynman diagrams for the problem under consideration to the
  desired accuracy.
\end{enumerate}
[seems out of place, as all steps except 5. are intractable by me]

\subsection{The Chiral Lagrangian}

In QCD the Lagrangian is\cite[p.~7]{Machleidt_2011}

\begin{equation*}
  \mathcal{L}_{\text{QCD}} = \bar{q}(i\gamma^{\mu}\mathcal{D}_{\mu} - \mathcal{M})q
  - \frac{1}{4}\mathcal{G}_{\mu\nu,a}\mathcal{G}^{\mu\nu}_{a}
\end{equation*}
with \(\mathcal{D}_{\mu} = \partial_{\mu} - i
g\frac{\lambda_{a}}{2}\mathcal{A_{\mu},a}\) the gauge-covariant derivative,
\(\mathcal{G}_{\mu\nu,a}\) the gluon field strength tensor, \(q\) the quark
fields and \(\mathcal{M}\) the quark mass matrix.

[vanishing quark mass]

[short about chiral symmetry]

[explicit and spontaneous symmetry breaking]

[chiral effective Lagrangian]


In QCD the u  and d quarks have approximately chiral symmetry.
Explicelty broken because u and d do not have the same mass
Spontaneously broken

Rekkeutvikling av Lpipi

Chiral order

\subsection{One Pion Exchange Force}

We obtain

\begin{equation*}
  V(\vec{r}) = \frac{f_{\pi}^{2}}{m_{\pi}^{2}}\left[ C_{\sigma}\sigma\cdot\sigma_{2} \right]
\end{equation*}

\begin{equation*}
  V^{LO}(\vec{q}, \vec{k}) = C_{s} + C_{t}\vec{\sigma_{a}}\cdot\vec{\sigma_{2}}
\end{equation*}

\begin{equation*}
  V^{NLO}(\vec{q}, \vec{k}) = C_{1}\vec{q}^{2} + C_{2}\vec{k}^{2} + \vec{\sigma}_{1}\cdot\vec{\sigma}_{2}
  + iC_{5}\frac{\vec{\sigma}_{1}+\vec{\sigma}_{2}}{2}\cdot\vec{q}\times\vec{k}
  + C_{6}\vec{q}\cdot\vec{\sigma}_{1}\vec{q}\cdot\vec{\sigma}_{2}
  + C_{7}\vec{k}\cdot\vec{\sigma}_{1}\vec{k}\cdot\vec{\sigma}_{2}
\end{equation*}

where \(\vec{q} = \vec{p} - \vec{p}^{\prime}\) is the momentum transfer, \(\k =
\frac{\vec{p}+\vec{p}^{\prime}}{2}\) the average momentum, and \(\vec{p}\) and
\(\vec{p}^{\prime}\) the relative momenta.


[Regularization]
\begin{equation*}
  V(p^{\prime}, p)
  \xlongrightarrow[\text{regularize}]{} f_{\Lambda}(p^{\prime})V(p^{\prime}, p)f_{\Lambda}(p)
\end{equation*}
where [why? Fourier transform trick]
\begin{equation*}
  f_{\Lambda}(p) = \exp{-p^{4}/Lambda^{4}}
\end{equation*}



%%% Local Variables:
%%% mode: latex
%%% TeX-master: "../main"
%%% TeX-engine: xetex
%%% End:
