\section{Introduction}

\epigraph{[It was] a whole new world. It was as if, suddenly, we had broken into
  a walled orchard, where protected trees flourished and all kinds of exotic
  fruits had ripened in great profusion.}{\textit{Cecil F. Powell \\ on the
    discovery of the pion\cite{powell}}} 

\subsection{Early Models on the Nuclear Force}

The onset of quantum theory in the 1920s laid the foundation for
understanding the interactions between nucleons from first principles. There
were difficulties in explaining how the nucleus kept together under the
repulsion between the protons, and in developing a formalism to understand the
nucleus. A
breakthrough came in 1935 when Hideki Yukawa proposed his meson theory,
where he used then-modern field calculations to predict the
existence of mesons\cite{Yukawa}. He proposed an attractive force between the nucleons
mediated by mesons, stemming from the residual strong force of the underlying quark
structure\footnote{This is a modern interpretation as Yukawa's paper was more
  focused on applying field theory to compute the mass of the meson(s).}. 
The success of Yukawa theory was solidified by the
later discovery of the \(\pi\)-meson\cite{LATTES1947} and heavier (\(\rho\), \(\omega\),
...) mesons\cite{PhysRevLett.6.628}.

Later work on pion theories in the 1950s had modest success with explaining
nucleon-nucleon (NN) scattering and properties of the deuteron involving one-pion
exchange, while multiple-pion exchange failed to produce reasonable
results\cite[p.~4]{10.1143/PTP.16.604,MACHLEIDT19871}. The discovery of heavier mesons during the 1960s motivated
one boson exchange models (OBE), models that still are of use today. There
were a plethora of nucleon potentials developed on these ideas (Stony
Brook\cite{JACKSON1975397}, Paris\cite{PhysRevC.21.861}, Bonn\cite{MACHLEIDT19871}), and the nucleon
force problem appeared to have been satisfyingly solved.

Alas, the development of quantum chromo-dynamics (QCD) in 1970s revealed that
the mesons themselves could be derived from more fundamental principles, and the work on
establishing the theoretical framework for nuclear interactions from first principles started 
all over again. Taking the low energy limit of QCD and exploiting chiral
symmetry lead to chiral perturbation theory (ChPT), an effective field theory which has achieved unprecedented
accuracy in modeling low energy nuclear potentials.

\subsection{Overview}
The goal of this text is to calculate the phase shift of \(^{1}S_{0}\) partial
waves of \(np\)-scattering using potentials inspired by ChPT. The potentials
are fitted to low energy phase shifts of the Reid68 potential and compared to
its high energy phase shifts.

We will begin with a rudimentary exposition of ChPT and present the nuclear
potentials that we will examine: LO, NLO and NNLO. Their form is sightly changed
by only including a one-pion exchange term modeled after the heaviest Yukawa
term of the Reid68 potential, thereby being simpler than the proper ChPT
potentials. In the methods section their numerical implementation are 
presented, and a fitting procedure is developed to fit the potentials to low
energy phase shifts. This turns out to be a tricky affair, requiring some care. Finally, phase shifts
are computed for larger energies and compared to the ``true'' phase shifts from
Reid68 and results of Lepage.


%%%   Local Variables:
%%%   mode: latex
%%%   TeX-master: "../main"
%%%   TeX-engine: xetex
%%%   End:
