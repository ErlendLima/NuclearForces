%\abstractintoc % Add abstract to Table of Contents 
%\abstractnum   % Format abstract like a chapter
                % Remove if abstract should not be on its own page

\makeatletter
\renewenvironment{abstract}{%
    \if@twocolumn
      \section*{\abstractname}%
    \else %% <- here I've removed \small
      \begin{center}%
        {\bfseries \large\abstractname\vspace{\z@}}%  %% <- here I've added \Large
      \end{center}%
      \quotation
    \fi}
    {\if@twocolumn\else\endquotation\fi}
\makeatother

\begin{abstract}

  Basic effective field theory is presented and used to derive nuclear
  toy potentials for \(np\)-scattering of \(^{1}S_{0}\) partial waves. The
  Reid68 potential is used to fit the potentials in the
  \mbox{\(10^{-3}-10^{-1}\) MeV} lab energy range, and compared in the
  \mbox{\(10^{-3}-10^{2}\) MeV} range. Other regions of fit were explored,
  confirming that the more infrared region, the better, but the
  aforementioned ranges were chosen to agree with Lepage. The fitting procedure
  was met with several difficulties, leading to poor fits and time consuming computations.
  Different values of the cutoff parameter
  \(\Lambda\) were explored, but no systematic power-law improvement
  improvement was seen as would be expected by chiral power counting. The
  closest improvements was found for \(\Lambda=100\) MeV for pionless potentials
  and \(\Lambda=150\) MeV when including \(1\pi\) term. The cause
  of this discrepancy was not identified and remains unresolved. The pionless potentials LO and NNLO
  obtained their minima below \(\Lambda = m_{\pi}\), indicating that the
  inclusion of pions is necessary to model higher energy physics. Including a
  one pion exchange toy potential based on the largest Yukawa term in the Reid68
  potential indeed increased the optimal value of \(\Lambda\) to about \(270\)
  MeV, in accordance with the results of Lepage.
\end{abstract}

%%% Local Variables:
%%% mode: latex
%%% TeX-master: "../main"
%%% TeX-engine: xetex
%%% End: