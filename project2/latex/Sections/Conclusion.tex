\section{Conclusion}\label{sec:Conclusion}

A basic overview of chiral perturbation theory was presented, casting low energy
nuclear physics as the low energy limit of QCD. Unlike pre-QCD and
phenomenological models, ChPT provides a rigorous treatment of the error, allowing
for systematic improvement. Its
faults lie in its construction, being only valid at low energies.

Several toy potentials inspired by effective field theory were constructed and used to model
\(np\)-scattering of \(^{1}S_{0}\) partial waves. They were fitted to low energy
phase shifts generated by the traditional Reid potential, and compared to phase shifts at
higher energies. The fitting procedure was fraught with problems, both
theoretical and practical, having difficulties in obtaining good fits and
convergence. The results showed consistent yet marginal improvement as higher
order terms were added, not in line with the improvement expected from power counting
arguments, and in disagreement with the results of Lepage. Poor fitting and too
simplistic models were suggested as causes, but none were satisfying in
explaining the disagreement.

EFT models involving a pionic term is expected to perform better than
pionless, and this was borne out by the calculations. Pionless models gave best
results right below \(\Lambda = m_{\pi}\), while the inclusion of the \(1\pi\)
toy term yielded best result near \(\Lambda = 270\) MeV, in agreement with
Lepage.



%%% Local Variables:
%%% mode: latex
%%% TeX-master: "../main"
%%% TeX-engine: xetex
%%% End: