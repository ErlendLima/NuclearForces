\message{ !name(../main.tex)}%-------------------- begin preamble ----------------------
%\documentclass[12pt]{article}
\documentclass[12pt, a4paper, abstracton]{scrartcl}
%\setkomafont{disposition}{\normalfont\bfseries}
\usepackage{Setup/style}
%---------------- custom style -----------------------
%----- equation numbering -------
\numberwithin{equation}{section}     % number within section
%\numberwithin{equation}{subsection} % number within subsection

%----- fig & table numbering -----
\counterwithin{figure}{section}
\counterwithin{table}{section}

%-------- header & foot -------- 
% FILL IN APPROPRIATE DESCRIPTIONS
% title page
\pagestyle{fancy}
\fancypagestyle{firststyle}
{
   \fancyhf{}
   \fancyfoot[C]{
   GitHub repository: 
   \url{https://github.com/ErlendLima/NuclearForces} 
   }
   \renewcommand{\headrulewidth}{0pt}
   \renewcommand{\footrulewidth}{0.4pt}
}
% rest of document
\fancyhead[L]{\small Nuclear Forces} 
\fancyhead[C]{\small Project 1} 
\fancyhead[R]{\small October 1, 2020}
\fancyfoot[C]{-- \thepage\ --}
\renewcommand{\headrulewidth}{0.4pt}
\renewcommand{\footrulewidth}{0.4pt}

%---------- insert custom LaTeX commands ----------
\newcommand*{\QED}{\hfill\ensuremath{\square}}
%usage: \QED
\newcommand{\BigO}[1]{\ensuremath{\operatorname{O}\bigl(#1\bigr)}}
%usage: \BigO{foo}
\newcommand*\rfrac[2]{{}^{#1}\!/_{#2}} 
%running fraction with slash. usage: $\rfrac{a}{b}$

%----------- New Commands --------------
\renewcommand{\thefootnote}{\fnsymbol{footnote}}
%%%%%%%%%%%%%%%%%%%%%%%%%%%%%%%%%%%%%%%%%%%%%%%%


\addbibresource{bibliography.bib}   % Bibliography
 %---------- included files ------------------------
\includeonly
{
    Sections/Frontpage,
    Sections/Abstract,
    Sections/Introduction,
    Sections/Theory, 
    Sections/Methods,
    Sections/Results,
    Sections/Discussion,
    Sections/Conclusion,
    %Sections/Future,
}

%-------------------- end preamble ----------------------

%-------------------- main content ----------------------
\begin{document}

\message{ !name(Sections/Theory.tex) !offset(-27) }
\section{Theory}\label{sec:Theory}

\newcommand{\pin}{\psi_\text{in}}
\newcommand{\pout}{\psi_\text{out}}
Quantum scattering is a very complicated process, behaving so differently from the
classical case of billiard balls that our intuition breaks down. Fortunately, the intricacies of
the scattering process itself can be conveniently omitted by instead focusing on
the relationship between the initial and resulting states.
Specifically, the true state \(\ket{\psi}\) is related to the asymptotic
incoming and outgoing states \(\ket{\psi_{}}\)
asymptotic free motion \(\ket{\chi}\) of a particle leaving an interaction and 
asymptotic free initial state \(\ket{\psi}\) 

\begin{equation*}
 w(\psi \to \chi) = \left| \mel{\chi}{\mathcal{S}}{\psi} \right|
\end{equation*}

\subsection{The Yukawa Potential}
[HISTORY]

\subsubsection{Momentum Basis}

Going from position basis to momentum space requires a change of basis, achieved
through a Fourier transform. The problem is multidimensional, requiring a
generalized Fourier transform. Since the Yukawa potential is radially 
symmetric, the Hankel transform can be used:

\newcommand{\kp}{k^\{\prime\}}
\begin{equation*}
  V_{l}(k, \kp) = \int_{0}^{\inf} \d r r^{2}
\end{equation*}

expressing the potential as a
weighted s

\subsection{The Lippman-Schwinger Equation}


\subsection{R-Matrix}\label{sec:Rmatrix}





%%% Local Variables:
%%% mode: latex
%%% TeX-master: "../main"
%%% End:

\message{ !name(../main.tex) !offset(-11) }

\end{document}
% ------------------- end of main content ---------------
