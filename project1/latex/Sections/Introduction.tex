\section{Introduction}
The interactions between nucleons are extremely complex and not yet fully
understood. As the quarks and gluons dance together within nuclei, their
strong interaction leaks out as the residual nuclear force, binding the nucleons
together.  In contrast to the electromagnetic force, the quantum field theory of the
strong force does not allow itself for practical computation of the resulting
nuclear force. As such, the nucleon-nucleon interaction must be described by
approximations and phenomenological models.

The phase shift is a useful proxy for understanding nucleon-nucleon interaction,
acting as a bridge between empirically obtained cross sections and theoretical
models.
Among these theoretical tools to compute the phase section are the \textit{K-matrix theory} and the \textit{variable
  phase approach}. Both of these are herein investigated by applying them
to a proton-neutron system as well as the square well.

The methods are derived through the
presentation of basic scattering theory. The scattering operator \(\mathcal{S}\) and
its relationship to the phase shift are introduced before moving on to informally
deriving the Lippmann-Schwinger equations yielding the equation for the
\(K\)-matrix. The regular solutions are then presented and used
to derive the variable phase approach, Levinson's theorem, and explain several aspects of the phase shift.

The theory is written to be self-contained, with emphasis on the intuition
instead of rigor. It mostly follows the excellent book of Taylor\cite{taylor},
but draws from multiple references. The numerical methods are implemented in
the~\href{https://julialang.org/}{Julia language}. All implementations as well
as examples of their usage are available at~\url{https://github.com/ErlendLima/NuclearForces}.

% When two nucleons interact, they may either enter a bound state or scatter.
% Though both of them are described by the Schr\"{o}dinger equation, scattering
% states are more conveniently described by the \textit{Lippmann-Schwinger
%   equation}. 

%%%   Local Variables:
%%%   mode: latex
%%%   TeX-master: "../main"
%%%   TeX-engine: xetex
%%%   End:
