\section{Introduction}
The interactions between nucleons is [fullfør]
In contrast to the electromagnetic force, there is no fundamental theory of the
strong force. As the quarks and gluons dance together within nuclei, their
strong interaction leaks out as the residual nuclear force, binding the nucleons
together. As such, the nucleon-nucleon interaction must be described by
approximations and phenomenological models.

The phase shift is a useful proxy for understanding nucleon-nucleon interaction,
acting as a bridge between empirically obtained cross sections and theoretical
models.
Among these theoretical tools to compute the phase section are the \textit{K-matrix theory} and the \textit{variable
  phase approach} (VPA). Both of these are herein investigated by applying them
to a two-nucleon system, implemented in the~\href{Julia language}{https://julialang.org/}.

% When two nucleons interact, they may either enter a bound state or scatter.
% Though both of them are described by the Schr\"{o}dinger equation, scattering
% states are more conveniently described by the \textit{Lippmann-Schwinger
%   equation}. 
