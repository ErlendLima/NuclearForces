\section{Theory}\label{sec:Theory}

\newcommand{\state}{\ket{\psi}}
\newcommand{\hoket}{\ket{\phi}}
\newcommand{\hoep}{\ket{\phi +}}
\newcommand{\hoem}{\ket{\phi -}}
\newcommand{\pin}{\psi_\text{in}}
\newcommand{\pout}{\psi_\text{out}}
\newcommand{\kpin}{\ket{\psi_\text{in}}}
\newcommand{\kpout}{\ket{\psi_\text{out}}}
\newcommand{\mop}{\Omega_{+}}
\newcommand{\mom}{\Omega_{-}}
\newcommand{\melt}[3]{\left|\mel{#1}{#2}{#3}\right|^{2}}
\newcommand{\scat}{\mathcal{S}}
\newcommand{\mscat}{\(\scat\)}
\newcommand{\oop}[1]{\mathcal{#1}}
\newcommand{\moop}[1]{\(\oop{#1}\)}
Quantum scattering is a very complicated process, behaving so differently from the
classical case of billiard balls that our intuition breaks down. Fortunately, the intricacies of
the scattering process itself can be conveniently omitted by instead focusing on
the relationship between the initial and resulting states, illustrated in FIGURE.
Specifically, the true state \(\ket{\psi}\) is related to the asymptotic
incoming and outgoing states \(\kpin\) and \(\kpout\) through the so-called
\textit{M\o ller operators} \(\Omega_{\pm}\):

\begin{align*}
  \state &= \mop\kpin = \hoep\\
  \state &= \mom\kpout = \hoem .
\end{align*}
[Add some history of Møller]

If time dependence is added, the M\o{}ller operations allows one to move between
the asymptotic states and the actual state at time \(t\).
Combining them, the outgoing state is related to the incoming state by

\begin{equation*}
  \kpout = \mom^{\dagger}\mop\kpin = \scat\kpin
\end{equation*}

giving the definition of the \textit{scattering operator} \mscat.
In the general case let \(\ket{\chi -}\) and \(\ket{\Phi +}\) by any arbitrary
orbits. The probability of the process \(\ket{\chi -}\leftarrow\ket{\Phi +}\)
occurring is
then the square of the  matrix element of\ \mscat :

\begin{equation*}
  w(\chi \leftarrow \Phi) = \left| \ip{\chi -}{\Phi +} \right|^{2} = \melt{\chi}{\scat}{\Phi}.
\end{equation*}

The probability \(w\) is itself not observable, however, the related quantity
cross-section \mbox{\(\sigma(\chi \leftarrow \Phi)\)} is. The outgoing particle can
scatter in a solid angle \(d\Omega\), oriented in the direction of momentum
\(\vec{p}\). Likewise, the incoming particle can be described as wave packets
with narrowly defined momentum \(\vec{p_{0}}\). It can then be shown that[CITE] the
differential cross section is

\begin{align*}
  \dv{\sigma}{\Omega}(\vec{p}\leftarrow\vec{p_{0}}) = \left| f(\vec{p}\leftarrow\vec{p_{0}}) \right|^{2}
\end{align*}
with \(f(\vec{p}\leftarrow \vec{p_{0}})\) being the scattering amplitude, as known from elementary scattering
theory. Note that only the magnitude of \(f\) can be obtained through the
cross-section.

The next step is to relate cross-section to the scattering operator. It
can be shown that \mscat{} commutes with the free Hamiltonian\( \oop{H}_{0}\)
\begin{align*}
  \oop{H} = \oop{H}_{0} + V &\qquad [\oop{H}_{0}, \scat{}] = 0
\end{align*}
Letting \(\ket{\vb{p}}\) be the eigenvectors of \(\oop{H}_{0}\) in momentum
basis, we have
\begin{align*}
  \mel{\vb{p}^{\prime}}{[\oop{H}_{0}, \scat]}{\vb{p}} = (E_{p^{\prime}}-E_{p})\mel{\vb{p}^{\prime}}{\scat}{\vb{p}} = 0
\end{align*}
implying that \(\mel{\vb{p}^{\prime}}{\scat}{\vb{p}}\) is zero except for
\(E_{p^{\prime}}=E_{p}\). This leads to the form
\begin{align*}
  \mel{\vb{p}^{\prime}}{\scat}{\vb{p}} = \delta(E_{p^{\prime}}-E_{p})\times\text{remainder}
\end{align*}

At this point it is fruitful to define a new operator \(\oop{R} \equiv 1 -
\scat\). \moop{R} is the difference between the case of scattering and no scattering. It too commutes with \(\oop{H}_{0}\), and so has the form
\begin{align*}
  \mel{\vb{p}^{\prime}}{\oop{R}}{\vb{p}} = -2\pi i\delta(E_{p^{\prime}}-E_{p})t(\vb{p}^{\prime}\leftarrow \vb{p})
\end{align*}
with the factors \(-2\pi i\) and \(t(\vb{p}^{\prime}\leftarrow\vb{p})\)
introduced for future convenience. The elements of\ \mscat{} can therefore be written 
\begin{align*}
  \mel{\vb{p}^{\prime}}{\scat}{\vb{p}} = \delta(\vb{p}^{\prime} - \vb{p}) - 2\pi i \delta(E_{p^{\prime}}-E_{p})t(\vb{p}^{\prime}\leftarrow \vb{p}).
\end{align*}
The first term describes the situation where no scattering occurs, while the
second is the amplitude when the wave is actually scattered. The function
\(t(\vb{p}^{\prime}\leftarrow \vb{p})\) is continuous for most potentials and
analytic for many, but only defined for the ``shell'' \(\vb{p}^{{\prime}^{2}} =
\vb{p}^{2}\). However unphysical, it is computationally beneficial to define an
operator \moop{T} whose matrix elements \(\mel{\vb{p}^{\prime}}{T}{\vb{p}}\) are defined for all \(\vb{p}\) and
coincide with the values of \(t\) on the shell.

The scattering amplitude can be shown to be related to the on-shell \moop{T}
matrix elements as[cite]
\begin{align*}
  f(\vb{p}^{\prime} \leftarrow \vb{p})= -(2\pi)^{2}mt(\vb{p}^{\prime}\leftarrow\vb{p}).
\end{align*}
The elements of the scattering matrix can from this be directly related to the
scattering amplitude as
\begin{align*}
  \mel{\vb{p}^{\prime}}{\scat}{\vb{p}} = \delta(\vb{p}^{\prime} - \vb{p}) - \frac{i}{2\pi m}\delta(E_{p^{\prime}}-E_{p})f(\vb{p}^{\prime}\leftarrow \vb{p}).
\end{align*}

\subsection{The Spherical Case}

Introducti

As known from elementary quantum mechanics, \(\oop{H}^{0}\) commutes with the
angular momentum operator \(\oop{L}^{2}\) and z-axis projection \(\oop{L}_{3}\).
These three operators form a complete set of commuting observables. Since
\mscat{} commutes with \(\oop{H}^{0}\), \mscat{} too commutes with the
aforementioned 
operators, and is diagonal in the common basis, namely the basis of spherical waves
\(\left\{ \ket{Elm} \right\}\), with \(E, l(l+1)\) and \(m\) being the
eigenvalues for \(\oop{H}^{0}\),  \(\oop{L}^{2}\) and  \(\oop{L}_{3}\)
respectively. Since this basis diagonalizes\ \mscat{}, its matrix elements are
\begin{align*}
  \mel{E^{\prime}l^{\prime}m^{\prime}}{\scat}{Elm} = \delta(E^{\prime}-E)\delta_{l^{\prime}l}\delta_{m^{\prime}m}s_{l}(E).
\end{align*}

\mscat{} can be shown to be unitary, implying the eigenvalues \(s_{l}(E)\) must have
modulus one, justifying the form
\begin{align*}
  \mel{E^{\prime}l^{\prime}m^{\prime}}{\scat}{Elm} = \delta(E^{\prime}-E)\delta_{l^{\prime}l}\delta_{m^{\prime}m}\exp(2i\delta_{l}(E))
\end{align*}
The quantity \(\delta_{l}(E)\) is the important \textit{phase shift}, an
observable obtainable from both experiment and numerical calculations. It is
purely real with an inherent ambiguity modulo \(\pi\), as seen from
\begin{align*}
  \exp(2i[\delta_{l}+n\pi]) = \exp(2i\delta_{l})\exp(2in\pi) = \exp(2i\delta_{l}).
\end{align*}

The decomposition of \(f(\vb{p}^{\prime}\leftarrow\vb{p})\) into partial waves
can be obtained by exploiting the relation

\begin{equation}
  \label{eq:sp1}
  \mel{\vb{p}^{\prime}}{(\scat-1)}{\vb{p}} = \frac{i}{2\pi m}\delta(E^{\prime}-E)f(\vb{p}^{\prime}\leftarrow \vb{p}).
\end{equation}
Inserting a complete set of states of the left hand side gives
\begin{align*}
  \mel{\vb{p}^{\prime}}{(\scat-1)}{\vb{p}} &= \int \dd E \sum_{l, m}  \mel{\vb{p}^{\prime}}{(\scat-1)}{Elm}\ip{Elm}{\vb{p}}\\
                                             &=  \int \dd E \sum_{l, m} (s_{l}(E)-1) \ip{\vb{p}^{\prime}}{Elm}\ip{Elm}{\vb{p}}\\
  &= \frac{1}{mp}\delta(E_{p^{\prime}}-E_{p})\sum_{l,m}Y_{l}^{m}(\vu{p}^{\prime})[s_{l}(E_{p})-1]Y^{m}_{l}(\vu{p})^{*}
\end{align*}
where \(Y_{l}^{m}\) is the spherical harmonical and hat denotes unit vector. Combining this
with~\eqref{eq:sp1}, the amplitude can be decomposed to

\begin{align*}
  f(\vb{p}^{\prime}\leftarrow\vb{p}) = \frac{2\pi}{ip}\sum_{l,m}Y_{l}^{m}(\vu{p}^{\prime})[s_{l}(E_{p})-1]Y^{m}_{l}(\vu{p})^{*}
\end{align*}
Letting \(\vu{p}\) lie along \(z\) and noting independence of \(m\) [WHY], we
define
\begin{align*}
  f(E_{p},\theta) \equiv f(\vb{p}^{\prime}\leftarrow\vb{p})=\frac{1}{2ip}\sum_{l}(2l+1)[s_{l}(E_{p})-1]P_{l}(\cos\theta)
\end{align*}
This leads to the natural definition of the \textit{partial wave amplitude} as
\begin{align*}
  f_{l}(E) \equiv \frac{s_{l}(E)-1}{2ip} = \frac{\exp[ 2i\delta_{l}(\theta)] - 1}{2ip} = \frac{\exp[2i\delta_{l}(E)]\sin\delta_{l}(E)}{p}
\end{align*}
Analogously the total cross-section can be decomposed into \textit{partial wave
  cross-sections}, giving
\begin{align*}
  \sigma(p) = \sum_{l}\sigma_{l}(p) = \sum_{l}4\pi(2l+1)\left| f_{l}(p) \right|^{2} = \sum_{l}4\pi(2l+1)\frac{\sin^{2}\delta_{l}}{p^{2}}
\end{align*}
The magnitude of each partial wave cross-section is from this constrained by the
so called \textit{unitary bound}: 
\begin{align*}
  |\sigma_{l}| \leq 4\pi\frac{2l+1}{p^{2}}.
\end{align*}
The maximal value is only reached if
\(\delta_{l}\) is an odd multiple of \(\pi/2\)

\subsection{Green's Function}
[Introduce motivation for free and full Green]

[Trivially derive LS]

\subsection{Lippman Schwinger}
[Use LS for G to derive it for T]

\subsection{The K-Matrix}
[This section was written before I understood the connection between K and T.
Use it to motivate the introduction and resulting computation of K]

To actually obtain the phase shifts numerically, we turn to yet another matrix,
the \(K\)-matrix\footnote{The nomenclature of the \(K\)-matrix is a mess. It is
  also called the \(R\)-matrix, the \(T\)-matrix, the reaction matrix, the
  reactance matrix, the dampening matrix, the distortion matrix, and Heitler's
  matrix. To avoid overworking any of the other
  letters, I follow the style of Taylor.}. Its associated operator\ \moop{M} is the Caley transform of
\moop{S}, ensuring\ \moop{M} is Hermitian:

\begin{equation*}
  \oop{M} = i \frac{1-\oop{S}}{1+\oop{S}}.
\end{equation*}
The \(K\)-matrix is defined as the matrix elements of\ \moop{M}
\begin{equation*}
  \mel{\vb{p}^{\prime}}{\oop{M}}{\vb{p}} = \delta(E_{p^{\prime}}-E_{p})k(\vb{p}^{\prime}\leftarrow\vb{p}).
\end{equation*}
When\ \moop{S} is symmetric,\ \moop{M} is both Hermitian and
symmetric, implying its matrix elements are real.
As with\ \moop{S} and\ \moop{R}, \moop{M} becomes diagonal in the basis
\(\left\{ \ket{Elm} \right\}\):
\begin{align*}
  \mel{E^{\prime}l^{\prime}m^{\prime}}{\oop{S}}{Elm} &= \delta(E_{p^{\prime}}-E_{p})\delta_{l^{\prime},l}\delta_{m^{\prime},m}s_{l}(E)\\
  \mel{E^{\prime}l^{\prime}m^{\prime}}{\oop{R}}{Elm} &= \delta(E_{p^{\prime}}-E_{p})\delta_{l^{\prime},l}\delta_{m^{\prime},m}2ipf_{l}(E)\\
  \mel{E^{\prime}l^{\prime}m^{\prime}}{\oop{M}}{Elm} &= \delta(E_{p^{\prime}}-E_{p})\delta_{l^{\prime},l}\delta_{m^{\prime},m}k_{l}(E).\\
\end{align*}
The elements \(k_{l}(E)\) will in this case be real, and are directly related to
the phase shifts:
\begin{align*}
  k_{l} = i \frac{1-s_{l}}{1+s_{l}} = \tan\delta_{l}.
\end{align*}

[fix factor 1/p]

%The \(K\)-matrix is to\ \moop{M} as \(T\) is to\ \moop{R}.
It can be shown that in the same way as\ \moop{S} is related to expansion into
plane wave stationary states \(\ket{\vb{p}\pm}\),\ \moop{M} is related to expansion into standing
waves \(\ket{\vb{p}s}\)

[Expand a bit on standing waves]

[Derive Heitler's equation]
\subsection{Bound States and Levinson's Theorem}
[The Jost function]
[Poles of S]
[Levinson]

\subsection{Potentials}
\subsubsection{The Square Well}
Near and dear to everyone is the square well potential; a simple discontinuous
potential with value \(V\_{0}\) in the interacting region and zero everywhere
else:
\begin{equation*}
  V(r) =
  \begin{cases}
    V_{0} & \text{for } 0 \leq r \leq R\\
    0 & \text{for } r > R 
  \end{cases}
\end{equation*}

The potential is plotted in~\cref{fig:squarewell}

\begin{figure}[H]
  \centering
  \includegraphics[]{Figures/squarewell.pdf}
  \caption{\label{fig:squarewell} An attractive square well potential with
    \(V_{0}=-1\) MeV.}
\end{figure}

[Analytical Solutions]


\subsubsection{The Yukawa Potential}
[HISTORY]

The Yukawa potential was later generalized into a class of \textit{generalized
  Yukawa potentials}. They are potentials build on superpositions of Yukawa potentials:
\begin{equation*}
  V(r) = \sum_{i=1}^{N}C_{i}\frac{e^{-\eta_{i}r}}{r}
\end{equation*}
for some coefficients \(C_{i}\) and \(\eta_{i}\).

A specific instance of a generalized Yukawa potential is the \textit{Reid potential}. It
is a parameterized potential between a proton and a neutron for the partial wave
\(^{1}S_{0}\), consisting of three terms:
\begin{equation*}
  V(r) = V_{a}\frac{e^{-ax}}{x} + V_{b}\frac{e^{-bx}}{x} + V_{c}\frac{e^{-cx}}{x}
\end{equation*}
where \(x=\mu r\), \(\mu=0.7\) MeV, \(V_{a}=-10.463\) MeV, \(V_{b}=-1650.6\)
MeV, \(V_{c}=6484.3\) MeV, and \(a=1\), \(b=4\) and \(c=7\).
[History of Reid].

\subsubsection{Momentum Basis}

Going from position basis to momentum space requires a change of basis, achieved
through a Fourier transform. The problem is multidimensional, requiring a
generalized Fourier transform. As we are only dealing with spherically symmetric
and central potentials, the more suitable Hankel transform\footnote{The Hankel
  transform can be regarded as the Fourier transform in hyperspherical coordinates, expanding the function in
Bessel functions instead of sines and cosines. } can be used. For
\(s\)-waves, it takes the form:

\newcommand{\kp}{k^{\prime}}
\begin{equation*}
  V_{l}(k, \kp{}) = \int_{0}^{\infty}j_{0}(kr)V(r)j_{0}(k^{\prime}r)r^{2}\dd r .
\end{equation*}

[Square well in momentum basis]

[Reid in momentum basis]




%%% Local Variables:
%%% mode: latex
%%% TeX-master: "../main"
%%% End:
