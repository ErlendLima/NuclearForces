%\abstractintoc % Add abstract to Table of Contents 
%\abstractnum   % Format abstract like a chapter
                % Remove if abstract should not be on its own page

\makeatletter
\renewenvironment{abstract}{%
    \if@twocolumn
      \section*{\abstractname}%
    \else %% <- here I've removed \small
      \begin{center}%
        {\bfseries \large\abstractname\vspace{\z@}}%  %% <- here I've added \Large
      \end{center}%
      \quotation
    \fi}
    {\if@twocolumn\else\endquotation\fi}
\makeatother

\begin{abstract}

  Basic scattering theory is presented in an informal manner and used to derive
  two methods for computing the phase shift: by solving the integral equation of
  the \(K\)-matrix, and by solving the differential equation of the variable
  phase approach. The methods yield the same results, with the \(K\)-matrix
  being computationally less expensive. For the square well they agree with the
  analytical solutions. The Reid potential was used to model the
  \(^{1}S_{0}\) partial wave of neutron-proton scattering. When
  compared to the experimental data of the Nijmegen group, the general shape is reproduced. The discrepancies are believed to
  be inherent to the Reid potential itself, not the methods used.
\end{abstract}

%%% Local Variables:
%%% mode: latex
%%% TeX-master: "../main"
%%% TeX-engine: xetex
%%% End: