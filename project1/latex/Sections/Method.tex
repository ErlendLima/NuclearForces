\section{Method}\label{sec:Method}

\subsection{Discretizing Lippman-Schwinger}
\label{sec:discret}

The equation [REF] can be solved numerically by iteration. To this end it is
discretized through a series of steps. The principal value poses a problem due to the limited precision of computers.
To step around this, the integral is rewritten. The function
\(\frac{1}{k-k_{0}}\) is even about \(k_{0}\), hence having parts of the curve
above and below the \(x\)-axis, meaning

\begin{align*}
  \int_{-\infty}^{\infty}\frac{\dd k}{k-k_{0}} = 0.
\end{align*}
Breaking the integral into its positive and negative parts and performing the
substitution \(k\to -k\) in the negative part then yields
\begin{align*}
  \int_{0}^{\infty}\frac{\dd k}{k^{2}-k_{0}^{2}} = 0
\end{align*}

The principal value can now be expressed as

\begin{align*}
  \mathcal{P}\int_{0}^{\infty}\frac{f(k)}{k^{2}-k_{0}^{2}} \dd k = \int_{0}^{\infty}\frac{f(k)-f(k_{0})}{k^{2}-k_{0}}\dd k
\end{align*}
The integral is no longer singular at \(k_{0}\), instead being proportional to
\(\dv{f}{k}\), allowing it to be computed numerically.

In particular, [REF] can be recast to the form

\begin{align*}
  K(k, k^{\prime}) = V(k, k^{\prime}) + \frac{2}{\pi}\int_{0}^{\infty}
  \dd q \frac{q^{2}V(k,q)K(q,k^{\prime} - k^{2}_{0}V(k, k_{0})K(k_{0},k^{\prime})}{(k_{0}^{2}-q^{2})/m}
\end{align*}

The integral is approximated through
\begin{equation*}
  \int_{-1}^{1}f(x)\dd x \approx \sum_{i=1}^{N}f(x_{i})w_{i}
\end{equation*}
for N lattice points through the corresponding weights \(w_{i}\) and points
\(x_{i}\in [-1, 1]\). This is done through Gaussian-Legendre quadrature, using
Legendre polynomials to construct the weights and corresponding points. To map
the integral from \([-1, 1]\) to \([0, \inf]\), the points and weights are
transformed by

\begin{align*}
  k_{i} &= C\times \tan{\frac{\pi}{4}(1+x_{i})}\\
  w_{i} &= C\times \frac{\pi}{4}\frac{w_{i}}{\cos^{2}\left( \frac{\pi}{4}(1-x_{i}) \right)}
\end{align*}
where \(C\) is a constant to fix the units.

Applied to our problem, it is discretized into

\begin{align*}
  K(k, k^{\prime}) = V(k, k^{\prime}) + \frac{2}{\pi}\sum_{i=1}^{N}
  \frac{w_{j}k_{j}^{2}V(k, k_{j})K(k_{j},k^{\prime})}{(k_{0}^{2}-k_{j}^{2})/m}
  - \frac{2}{\pi}k_{0}^{2}V(k, k_{0})K(k_{0},k^{\prime})\sum_{n=1}^{N}
  \frac{w_{n}}{(k_{0}^{2}-k_{n}^{2})/m}.
\end{align*}

There are \(N\times N\) unknowns for \(K(k, k^{\prime})\), plus 1 for \(K(k_{0},
k_{0})\), for a total of \(N+1\) unknowns. They can be described by a single
extended \(K\)-matrix. Defining another matrix \(A\) as
\begin{align*}
  A_{i,j} &= \delta_{i,j} - V(k_{i}, k_{j})u_{k}\\
  \intertext{with}\\
            u_{j} &= \frac{2}{\pi} \frac{w_{j}k_{j}^{2}}{(k_{0}^{2}-k_{j}^{2})/m}
                    \qquad \text{for } j = 1, 2, \ldots, N \\
  u_{N+1} &= - \frac{2}{\pi}\sum_{j=1}^{N}\frac{k_{0}^{2}w_{j}}{(k_{0}^{2}-k_{j}^{2})/m}.
\end{align*}
The implementation of this main loop is shown in~\cref{list:kmatrix}

\begin{listing}
\begin{minted}[linenos,mathescape=true,fontsize=\tiny,breaklines,escapeinside=||]{julia}
function createA(V, k, ω, m)
    N = length(k)-1
    k₀ = k[end]
    A = diagm(0 => repeat([1.0,], N+1))
    uⱼ = 0.0

    @inbounds for j in 1:N+1
        uⱼ = 0.0
        if j ≠ N+1
            uⱼ = 2/π * ω[j]*k[j]^2/((k₀^2 - k[j]^2)/m)
        else
            for n in 1:N
                uⱼ += ω[n]*k₀^2/((k₀^2 - k[n]^2)/m)
            end
            uⱼ *= -2/π
        end

        for i in 1:N+1
            A[i, j] -= V[i, j] * uⱼ
        end
    end
    return A
end
\end{minted}
\label{list:kmatrix}
\caption{The main loop of the \(K\)-matrix method.}
\end{listing}

The equation can now be rendered as the matrix equation

\begin{align*}
  AK = V.
\end{align*}

Solving for \(K\) then simply amounts to inverting \(A\) and computing the
product

\begin{equation}
  K = A^{-1}V.\label{eq:matrixeq}
\end{equation}

As known from [REF], the diagonal of \(K\) is proportional to the phase shift
\begin{align*}
  K(k_{N+1}, k_{N+1}) = K(k_{0}, k_{0}) = -\frac{1}{mk_{0}}\tan\delta_{0}(k_{0})
\end{align*}

The number of mesh points \(N\) controls both the accuracy and the computation
resources, demanding more memory and time as \(N\) increases. Running over the
\((N+1)\times (N+1)\) matrices scales as roughly \(\mathcal{O}(N+1)^{2}\) in
both time and memory allocation. This is the expected behavior to compare to
when measuring its actual resource usage.

\subsection{Variable Phase Approach}
In contrast to the LS approach, implementing the variable phase approach is
straightforward. Julia has a very good differential equations solver
\href{https://diffeq.sciml.ai/stable/#Getting-Started:-Installation-And-First-Steps}{DifferentialEq.jl}\cite{rackauckas2017differentialequations}, which is used for the actual computation.
Since the limit in the relation \(\delta(k) =
\lim_{\rho\to\infty}\delta(k, \rho)\) can not be taken to infinity on a computer, 
\(\rho\) is instead substituted with a very large number, like \(5\). There is a
trade-off between accuracy and computation time, with smaller \(\rho\) giving
lesser accuracy while taking shorter time to compute, assuming the same step
length is being used. To find the optimal \(\rho\), several values were tried
and the result compared to a high \(\rho\).

The code implementation is shown in~\cref{lst:vpa}

\begin{listing}
\begin{minted}[linenos,mathescape=true,fontsize=\tiny,breaklines,escapeinside=||]{julia}
struct VPA <: Method
  rspan::Tuple{Float64, Float64}  # Range of the interaction
end
VPA() = VPA((1e-4, 15.0))


function (vpa::VPA)(k, m, V::Potential)
    m = m/2

    function dδdr(δ, p, r)
      -1.0/k * 2.0m * V(r) * sin(k*r + δ)^2
    end

    δ0 = 0.0

    prob = ODEProblem(dδdr, δ0, vpa.rspan)
    sol = solve(prob, maxiters=10000, reltol=1e-8)

    δ = sol.u[end]
end
  \end{minted}
  \caption{Implementation of VPA in Julia}
  \label{lst:vpa}
\end{listing}

\subsection{The Potentials}

Two potentials are examined: the square well and the Reid potential. The former
has analytical solutions and is relatively simple to understand, and will be
used as a check on the implementation of the methods and to illustrate concepts
discussed in the theory. The latter is more complicated, being a sum of three
Yukawa potentials with the intent on modeling NN-interactions, with coefficients
fitted to data. 

Their precise implementation is not interesting, but is available in the file
\textsf{potentials.jl}. 


%%% Local Variables:
%%% mode: latex
%%% TeX-master: "../main"
%%% TeX-engine: xetex
%%% End:
