\section{Conclusion}\label{sec:Conclusion}

The basics of scattering theory was developed in terms of the \(\oop{S}\)
operator, yielding a plethora of fundamental results regarding the phase shift,
\(K\)-matrix, bound states and Levinson's theorem. The theory gave birth to two
methods for computing the phase shift: finding the K-matrix through its
Lippmann-Schwinger equation, and solving the differential equation of the
variable phase approach. They were used to find the phase shift of square wells
and of the Reid potential modeling the \(s\)-wave scattering of a \(np\) system.
Both methods gave equal phase shifts, with the phase shift of the square well
matching the analytical solution. The phase shift of the Reid potential
generally matched the experimental data of the Nijmegen group, but with
discrepancies that can only be attributed to the Reid potential itself. Zero
energy resonances were observed in square well examples as well as in the Reid
potential, showing that the potential is on verge on allowing for a bound state.

%%% Local Variables:
%%% mode: latex
%%% TeX-master: "../main"
%%% TeX-engine: xetex
%%% End: